%%%%%%%%%%%%%%%%%%%%%%%%%%%%%%%%%%%%%%%%%%%%%%%%%%%%%%%%%%%%%%%%%%%%%%%%%%%%%%%%
%% Introduction: 引出问题背景 -> 现有方法局限 -> 我们的方案 -> 贡献点
%%%%%%%%%%%%%%%%%%%%%%%%%%%%%%%%%%%%%%%%%%%%%%%%%%%%%%%%%%%%%%%%%%%%%%%%%%%%%%%%
\section{INTRODUCTION}
\label{sec:introduction}

% 背景:VLN任务的重要性和应用场景 - 满意
Vision-and-Language Navigation (VLN) enables embodied agents to follow natural language instructions in complex, unstructured environments~\cite{VLN_servey1, VLN_servey2}.
It has become a mainstream paradigm in the field of embodied intelligence~\cite{VL_Nav, IROS_VLN}.
However, the computational demands of VLN models fundamentally conflict with the real-time navigation requirements of practical robotic deployment~\cite{HIAI}.

\begin{figure*}[!t]
\centering
\includegraphics[width=\textwidth]{fig/placeholder_double.png}
\caption{Overview Placeholder for Observation and Motivation in VLN token caching.}
\label{fig:analysis-overview}
\end{figure*}

% VLN加速SOTA:模型架构创新、压缩、运行时优化 - Sota已经讲好了
Existing acceleration approaches for VLN primarily follow three directions: efficient architectures, model compression, and runtime optimization.
Efficient architectures design lightweight networks to reduce inherent complexity~\cite{efficientvln,etp_r1, LOGNav}.
Compression techniques like quantization shrink model size while preserving predictive capabilities~\cite{MiniVLN, NaVILA}.
Runtime optimization dynamically reduces computations during inference without altering model parameters~\cite{NAPVLN, NavEE, Uni-NaVid}.
These methods have substantially improved VLN inference efficiency.

% 引出 Token caching,并介绍Token caching核心直觉:时序连贯性 + 检测-复用范式 - 合并,并去掉细节表达,highlevel化
Token caching has emerged as a highly promising training-free runtime optimization technique.
It exploits temporal coherence: background regions such as walls and floors change little across adjacent frames, so their visual tokens can be safely reused rather than recomputed~\cite{vlacache,stcache, token_sparsification, EgoPrune}.
Following a detect-and-reuse paradigm, existing methods selectively skip the computation of tokens identified as static or task-irrelevant, achieving substantial speedup with minimal accuracy loss in fixed-camera or static-scene settings~\cite{vlacache, token_merging, stcache, EgoPrune, token_sparsification, Learn_VLA_cache}.

% SOTA Gap:静态/固定假设在VLN中失效
However, existing token caching methods are built on a static-view assumption: patches at the same image position are expected to remain similar across adjacent frames~\cite{vlacache}.
This assumption breaks in VLN, where the agent continuously translates and rotates during navigation~\cite{View_Invariant_Learning}.
As a result, physically static objects can shift substantially in image coordinates, making position-wise matching underestimate reusable tokens.
In addition, navigation is instruction-conditioned and temporally staged.
A landmark that is critical before a turn can quickly become irrelevant after the agent passes it, so semantic relevance changes even when appearance is stable.
These two dynamics jointly limit the direct transfer of static-scene caching strategies to embodied navigation.

% Analysis → Insight:点明做了分析,概括发现,引出方法
To investigate these dynamics, we conduct an empirical analysis of visual token behavior and instruction-conditioned region relevance across VLN trajectories.
Our analysis reveals two consistent patterns: physically static regions frequently appear dissimilar across adjacent frames due to viewpoint changes during navigation, and the task relevance of landmark regions varies substantially over the course of a trajectory.

% 方法概述:VLN-Cache双重感知框架
Motivated by these findings, we propose \textit{VLN-Cache}, a dual-aware token caching framework for VLN.
For visual dynamics, we perform view-aware feature alignment before cross-frame matching, which recovers reusable tokens that would be missed by position-wise comparison.
For semantic dynamics, we introduce instruction-guided saliency monitoring that refreshes cached tokens when their task relevance changes.
To control the additional overhead, we further adopt an entropy-based layer-adaptive reuse schedule that allocates conservative reuse to sensitive layers and more aggressive reuse to stable layers.
The resulting framework delivers practical inference acceleration while preserving overall navigation performance.

% 贡献点
Our key contributions are as follows.
\begin{itemize}
\item We provide empirical evidence that static-scene caching assumptions fail in VLN due to viewpoint-induced mismatch and temporal semantic shift.
\item We present \textit{VLN-Cache}, a dual-aware framework that combines view-aware matching with instruction-guided semantic refresh, without architectural changes or retraining.
\item We design an entropy-based layer-adaptive reuse strategy to balance acceleration gain and computational overhead across transformer layers.
\end{itemize}

Experimental results demonstrate that \textit{VLN-Cache} achieves a speedup of up to xxx$\times$ on standard VLN benchmarks, while maintaining competitive navigation success rates, supporting practical real-time embodied deployment.
